% $Id: appendix.tex 346 2008-02-25 00:39:00Z Franz $
\appendix

\chapter{Usage Notes}%
\label{pyclips-unotes}

\section{Environments}%
\label{pyclips-unotes-env}

As seen in the detailed documentation, \pyclips{} also provides access
to the \emph{environment} features of CLIPS, through the
\class{Environment} class. \class{Environment} objects provide almost
all functions normally available at the top level of \pyclips{}, that is
importing \code{clips} into a Python script. \class{Environment} object
methods having the same name of functions found at the top level of
\pyclips{}, have the same effect of the corresponding function -- but
restricted to the logical environment represented by the object
itself\footnote{In fact, the Python submodule that implements the
\class{Environment} class is generated automatically: the process can be
examined by looking at the code in \file{setup.py} and
\file{clips/_clips_wrap.py}.}. Normally in a true CLIPS session users would
not use environments, so the concept of environment may, in many cases,
not be useful.

There are also functions (namely: \function{CurrentEnvironment()} and
\function{Environment.SetCurrent()}) that allow the user to switch
environment and to use top level functions and classes in any of the
created environments. This is useful in cases where environments are used,
because due to the double nature of CLIPS API (see \clipsapg{} about
\emph{companion functions} for standard API), objects defined in
environments have slightly different types than corresponding top level
objects -- since these types are \emph{environment-aware}. However
environment aware classes define exactly the same methods as the top
level counterparts, so their logical use is the same.

\begin{notice}
Please note that the \function{CurrentEnvironment()} function returns
a fully functional \class{Environment} object. However the system
prevents\footnote{Raising an appropriate exception.} invasive access via
\emph{environment-aware} functions to current \emph{environment}: user
code should \emph{always} use functions defined at module level to access
current \emph{environment}. The \class{Environment} methods become safe
as soon as another \class{Environment} has become current -- and in this
case its methods and properties will raise an error.
\end{notice}

\class{Environment}s are a limited resource: this is because it is
impossible in \pyclips{} to destroy a created \class{Environment}.
In order to reuse \class{Environment}s it may be useful to keep a
reference to each of them for the whole \pyclips{} session. If there
is an attempt to create more \class{Environment}s than allowed, an
exception is raised.



\section{Multiple Ways\label{pyclips-unotes-multiw}}

There is more than one way to use the \pyclips{} module, since it exposes
almost all the API functions of CLIPS, seen in fact as a library, to the
Python environment.

The module \pyclips{} provides some functions, that is \function{Build()}
and \function{Eval()}, that let the user directly issue commands in the
CLIPS subsystem from a Python program. In fact, the use of \function{Build()}
allows definition of constructs in CLIPS\footnote{Note that the
\function{Build()} function does not return any value or object, so you
will have to call \function{Find\emph{Construct}()} to find entities created
using the \function{Build()} function.}, and \function{Eval()} lets the user
evaluate values or call code directly in the subsystem. So for instance
rules can be built in \pyclips{} using

\begin{verbatim}
>>> import clips
>>> clips.Build("""
(defrule duck-rule "the Duck Rule"
   (duck)
   =>
   (assert (quack)))
""")
>>> clips.PrintRules()
MAIN:
duck-rule
\end{verbatim}

and evaluate a sum using

\begin{verbatim}
>>> n = clips.Eval("(+ 5 2)")
>>> print n
7
\end{verbatim}

Also, the user is allowed to call functions that do not return a value
using \function{Eval()}\footnote{There is a discussion about functions
that only have \emph{side effects} in CLIPS, such as \code{printout}, in
\clipstut{}, that is, the CLIPS tutorial.}, as in the following example:

\begin{verbatim}
>>> clips.Eval('(printout t "Hello, World!" crlf)')
>>> print clips.StdoutStream.Read()
Hello, World!
\end{verbatim}

There is another function, namely \function{SendCommand()}, that sends
an entire CLIPS command (it has to be a full, correct command, otherwise
\pyclips{} will issue an exception): as \function{Build()} it does not
return any value or object\footnote{Some information about the command
result can be retrieved reading the appropriate output streams.}.
However this can be particularly useful when the user needs to implement
an interactive CLIPS shell within an application built on \pyclips{}.
Unless the application is mostly CLIPS oriented (if for instance Python
is used just as a ``glue'' script language) probably the use of this
function has to be discouraged, in favour of the code readability that
-- at least for Python programmers -- is provided by the Python oriented
interface.

Using \function{SendCommand()} it becomes possible to write:

\begin{verbatim}
>>> import clips
>>> clips.SendCommand("""
(defrule duck-rule "the Duck Rule"
   (duck)
   =>
   (assert (quack)))
""")
>>> clips.SendCommand("(assert (duck))")
>>> clips.Run()
>>> clips.PrintFacts()
f-0     (duck)
f-1     (quack)
For a total of 2 facts.
>>> clips.PrintRules()
MAIN:
duck-rule
\end{verbatim}

The most important caveat about \function{SendCommand()} is that CLIPS
accepts some kinds of input which normally have to be considered
incorrect, and \pyclips{} does neither return an error value, nor raise
an exception: for instance, it is possible to pass a symbol to CLIPS to
the command line as in

\begin{verbatim}
CLIPS> thing
thing
\end{verbatim}

and in this case CLIPS ``evaluates'' the symbol, printing it to the
console as a result of the evaluation. \pyclips{} does not automatically
capture evaluation output, and just accepts a symbol (or other commands
that can be evaluated) as input without any production:

\begin{verbatim}
>>> import clips
>>> clips.SendCommand("thing")
\end{verbatim}

but, turning on the verbosity flag:

\begin{verbatim}
>>> clips.SendCommand("thing", True)
>>> print clips.StdoutStream.Read()
thing
\end{verbatim}

Of course, \pyclips{} complains if something incorrect is passed to the
\function{SendCommand()} function and raises an exception as previously
stated. However the exception is accompanied by a rather non-explanatory
text. The \var{ErrorStream} object should be queried in this case in order
to obtain some more information about the error:

\begin{verbatim}
>>> clips.SendCommand("(assert (duck)")	# no right bracket

Traceback (most recent call last):
  File "<pyshell#5>", line 1, in -toplevel-
    clips.SendCommand("(assert (duck)")	# no right bracket
  File ".../_clips_wrap.py", line 2575, in SendCommand
    _c.sendCommand(s)
ClipsError: C09: unable to understand argument
>>> print clips.ErrorStream.Read()

[PRNTUTIL2] Syntax Error:  Check appropriate syntax for RHS patterns.
\end{verbatim}

Obviously \function{SendCommand()} can lead to serious errors if not
used with some kind of interaction.

The point of this paragraph is that, for entity definition (evaluation
can only be performed using the \function{Eval()} or \function{Call()}
functions), the \pyclips{} module provides a full set of specific
\function{Build\emph{Entity}()} functions which also return appropriate
objects corresponding to specific entities. So, the task of building a
\emph{rule} in CLIPS (in fact, a \class{Rule} object in Python) could
preferably be performed directly using the \function{BuldRule()}
function, that is:

\begin{verbatim}
>>> clips.Clear()
>>> clips.Reset()
>>> r0 = clips.BuildRule("duck-rule", "(duck)", "(assert (quack))",
                         "the Duck Rule")
>>> print r0.PPForm()
(defrule MAIN::duck-rule "the Duck Rule"
   (duck)
   =>
   (assert (quack)))

>>> clips.PrintRules()
MAIN:
duck-rule
\end{verbatim}

thus with the same effect as with the \function{Build()} function, but
obtaining immediately a reference to the rule entity in CLIPS as a Python
object. Similar examples could be provided for the \function{SendCommand()}
function, using the appropriate constructs or commands that can be used
to achieve the same goals.

This allows the user to choose between at least two programming styles in
\pyclips{}: the former, more CLIPS oriented, relies heavily on the use of
the \function{Build()}, \function{Eval()} and \function{SendCommand()}
functions, and is probably more readable to CLIPS developers. The latter
is somewhat closer to Python programming style, based on the creation of
objects of a certain nature by calling specific Python functions. The
advice is to avoid mixing the two styles unless necessary, since it can
make the code quite difficult to understand.



\section{Python Functions in CLIPS\label{pyclips-unotes-pyfuncs}}

In \pyclips{} it is possible to execute Python functions from within
CLIPS embedded constructs and statements. This allows the extension of
the underlying inference engine with imperative functionalities, as well
as the possibility to retrieve information from the Python layer
asynchronously with respect to Python execution. Of course this
possibility enables some enhancements of the CLIPS environment, but
-- as a drawback -- it also opens the way to errors and misunderstandings.

Usage of Python external functions is fairly simple: the user should
register functions that will be called from within the CLIPS subsystem
in \pyclips{} using the \function{RegisterPythonFunction()} toplevel
function. If no alternate name for the function is specified, then the
Python name will be used\footnote{An example of function registration
has been provided in the introduction.}. If necessary, Python function
names can be deregistered using \function{UnregisterPythonFunction()} and
\function{ClearPythonFunctions()} utilities. Once a function is registered
it can be called from within the CLIPS engine using the following syntax:

\begin{verbatim}
    (python-call <funcname> [arg1 [arg2 [ ... [argN]]]])
\end{verbatim}

and will return a value (this allows its use in assignments) to the CLIPS
calling statement. In the call, \code{<funcname>} is a \emph{symbol}
(using a string will result in an error) and the number and type of
arguments depends on the actual Python function. When arguments are of
wrong type or number, the called function fails. Using the previously
illustrated \function{py_square} example, we have:

\begin{verbatim}
>>> clips.RegisterPythonFunction(py_square)
>>> clips.SetExternalTraceback(True)	# print traceback on error
>>> print clips.Eval("(python-call py_square 7)")
49
>>> print clips.Eval('(python-call py_square "a")')
Traceback (most recent call last):
  File ".../_clips_wrap.py", line 2702, in <lambda>
    f = lambda *args: _extcall_retval(func(*tuple(map(_cl2py, list(args)))))
  File "<pyshell#85>", line 2, in py_square
TypeError: can't multiply sequence to non-int
FALSE
>>> print clips.Eval("(python-call py_square 7 7)")
Traceback (most recent call last):
  File ".../_clips_wrap.py", line 2702, in <lambda>
    f = lambda *args: _extcall_retval(func(*tuple(map(_cl2py, list(args)))))
TypeError: py_square() takes exactly 1 argument (2 given)
FALSE
\end{verbatim}

It is important to know, in order to avoid errors, that the Python
interpreter that executes functions from within CLIPS is exactly the
same that calls the \pyclips{} function used to invoke the engine: this
means, for example, that a Python function called in CLIPS is subject
to change the state of the Python interpreter itself. Moreover, due
to the nature of CLIPS external function call interface, Python functions
called in CLIPS will never raise exceptions\footnote{Exceptions can
arise \emph{during} the Python function execution, and can be caught
inside the function code. However, for debugging purposes, there is the
possibility to force \pyclips{} print a standard traceback whenever an
error occurs in a Python function called by CLIPS.} in the Python calling
layer.

Here are some other issues and features about the nature of the external
function call interface provided by \pyclips{}:

\emph{Functions should be CLIPS-aware:} when CLIPS calls a Python external
function with arguments, these are converted to values that Python can
understand using the previously described \emph{wrapper classes}. Thus,
for instance, if the Python function is given an integer argument, then
an argument of type \class{Integer} (not \class{int}) will be passed as
actual parameter. This means that in most cases \pyclips{} has to be
imported by modules that define external Python functions.

\emph{Actual parameters cannot be modified:} there is no way to pass values
back to CLIPS by modifying actual parameters. The possibility to use
\class{Multifield} parameters as lists should not deceive the user, as
every modification performed on \class{Multifield}s that Python receives
as parameters will be lost after function completion. A way to handle this
is to treat parameters as \emph{immutable} values.

\emph{External functions should always return a value:} functions always
return a value in CLIPS, even in case of an error. This can be clearly
seen in the following chunk of CLIPS code:

\begin{verbatim}
CLIPS> (div 1 0)
[PRNTUTIL7] Attempt to divide by zero in div function.
1
\end{verbatim}

where, although an error message is printed to the console, the value
\constant{1} is returned by the system. In the same way, CLIPS expects
Python external functions to return a value. \pyclips{} solves this issue
by converting a return value of \constant{None} (which is the real return
value for Python functions that simply \code{return}) into the symbol
\code{nil}, that has a meaning similar to the one of \constant{None} for
Python. Also, functions that raise uncaught exceptions will in fact return
a value to the underlying CLIPS engine: in this case the returned value
is the symbol \code{FALSE}, and an error message is routed to the error
stream -- thus, it can be retrieved using \function{ErrorStream.Read()}.
The following example imitates the \code{div} CLIPS example above:

\begin{verbatim}
>>> import clips
>>> exceptor = lambda : 1 / 0
>>> clips.RegisterPythonFunction(exceptor, 'exceptor')
>>> clips.SetExternalTraceback(True)	# print traceback on error
>>> clips.Eval('(python-call exceptor)')
Traceback (most recent call last):
  File ".../_clips_wrap.py", line 2702, in <lambda>
    f = lambda *args: _extcall_retval(func(*tuple(map(_cl2py, list(args)))))
  File "<pyshell#79>", line 1, in <lambda>
ZeroDivisionError: integer division or modulo by zero
<Symbol 'FALSE'>
\end{verbatim}

\emph{Return values must be understood by CLIPS:} only values that can
be converted to CLIPS base types can be returned to the inference engine.
This includes all values that can be converted to \pyclips{} \emph{wrapper
classes}. In fact it can be considered a good practice to cast return
values to \pyclips{} \emph{wrapper classes} when the main purpose of a
function is to be called from within CLIPS.

\emph{Python functions act as generic functions:} due to the nature of
Python, functions are generally polymorphic:

\begin{verbatim}
>>> def addf(a, b):
        return a + b
>>> print addf("egg", "spam")
eggspam
>>> print addf(2, 4)
6
\end{verbatim}

The intrinsic polymorphism of Python functions is kept within the CLIPS
subsystem:

\begin{verbatim}
>>> import clips
>>> clips.RegisterPythonFunction(addf)
>>> print clips.Eval('(python-call addf "egg" "spam")')
eggspam
>>> print clips.Eval('(python-call addf 2 4)')
6
\end{verbatim}

Thus Python functions act in a way that is similar to \code{generic}s.



\chapter{The \module{clips._clips} Submodule\label{pyclips-llclips}}

It has been said throughout the whole document that there are two
different ``natures'' of \pyclips{}, called the \emph{high level module}
and the \emph{low level module}. The former has been described in detail
here, and is supposed to be the main interface to CLIPS. However, since
all communication between Python and the CLIPS subsystem is implemented
in the \emph{low level} part, some users might find it useful to access
this interface instead of the \emph{higher level} one. It is not the
intention of this manual to provide documentation for the \emph{low
level} \module{clips._clips} interface, but only to give some indications
to users who already have experience of the CLIPS C interface.

Submodule \module{clips._clips} provides low-level classes that have the
same names as their counterparts in CLIPS, such as \class{fact},
\class{deftemplate} and so on. Also, \module{clips._clips} defines an
\class{environment} class to refer to \emph{environment} addresses.

Almost all functions described in \clipsapg{} are ported to Python,
except for missing features described below: the name is the same as in
the reference guide, except for the first letter that is not
capitalized\footnote{There can be some exceptions: the most
important one is the assertion function, since Python already has an
\keyword{assert} keyword. This function is called \function{assertFact}
instead.}. CLIPS ``top level'' functions have been ported to
\module{clips._clips} as well as \emph{companion functions} that accept
an \class{environment} instance as the first argument\footnote{When
trying to show the \emph{documentation string} of these functions, the
first argument is not described because their code has been generated
automatically.}, whose name begins with \code{env_} followed by the same
name as the corresponding top level function.

\emph{Low level} functions are documented by themselves through
\emph{documentation strings}, that describe purpose, arguments and return
values. For instance, let's show the documentation of a function:

\begin{verbatim}
>>> import clips
>>> cl = clips._clips
>>> print cl.listDeftemplates.__doc__
listDeftemplates(logicalname [, module])
list deftemplates to output identified by logicalname
arguments:
  logicalname (str) - the logical name of output
  module (module) - the module to inspect, all modules if omitted
\end{verbatim}

Most low level function documentation strings, i.e. the ones given
for functions that are not trivially identifiable with the CLIPS API
counterparts\footnote{Some functions have a documentation string that
actually refers to the CLIPS API itself, explicitly containing the words
``\emph{equivalent of C API}'' and the C function name: \clipsapg{} is
especially useful in this case.}, have this form and describe in detail
arguments and return values. Users who want to benefit of these
functions, can display \emph{documentation strings} as a reference.

The underlying engine is the same, there is no separation of environment
between the two interfaces. Operations performed at \emph{lower level}
are obviously reflected in the \emph{higher level} layer, as in the
following example:

\begin{verbatim}
>>> clips.Clear()
>>> cl.facts("stdout")
>>> s = clips.StdoutStream.Read()
>>> print s
None
>>> print cl.assertString.__doc__
assertString(expr) -> fact
assert a fact into the system fact list
returns: a pointer to the asserted fact
arguments:
  expr (str) - string containing a list of primitive datatypes
>>> f = cl.assertString("(duck)")
>>> clips.PrintFacts()
f-0     (duck)
For a total of 1 fact.
\end{verbatim}

so the two interfaces can be used interchangeably.



\chapter{Error Codes}%
\label{pyclips-errors}

It has been discussed above, that some of the \pyclips{} functions can
raise a CLIPS specific exception, namely \exception{ClipsError}. Some
of the exceptions of this type (in fact, the ones raised by the
underlying CLIPS engine and caught at the lower level), come with an
error code in the accompanying text. A brief description of exceptions
that arise at low level follows:

\begin{tableii}{l|l}{code}{Code}{Description}
	\lineii{P01}{An object could not be created, due to memory
            issues}
	\lineii{C01}{The engine could not create a system object}
	\lineii{C02}{The referred object could not be found in
            subsystem}
	\lineii{C03}{An attempt was made to modify an unmodifiable
            object}
	\lineii{C04}{A file could not be opened}
	\lineii{C05}{The current environment could not be retrieved}
	\lineii{C06}{The engine was unable to return the required value}
	\lineii{C07}{Parse error in the passed in CLIPS file}
	\lineii{C08}{Syntax or parse error in the passed in CLIPS
            expression}
	\lineii{C09}{Syntax or parse error in the passed in argument}
	\lineii{C10}{Expression could not be evaluated, maybe because of
            syntax errors}
	\lineii{C11}{An object could not be removed from the CLIPS
            subsystem}
	\lineii{C12}{A fact could not be asserted, maybe because of
            missing \code{deftemplate}}
	\lineii{C13}{Iteration beyond last element in a list}
	\lineii{C14}{A CLIPS function has been called unsuccessfully}
	\lineii{C15}{Attempt to modify an already asserted fact was made}
	\lineii{C16}{Cannot destroy environment while it is current}
	\lineii{C90}{Other errors: specific description given}
	\lineii{C97}{The feature is present when a higher version of
            CLIPS is used}
	\lineii{C98}{An attempt was made to use an unimplemented feature
            (see below)}
	\lineii{C99}{Generic CLIPS error, last operation could not be
            performed}
	\lineii{C00}{Generic CLIPS error, no specific cause could be
            reported}
        \lineii{S01}{Attempt to access a fact that is no longer valid
            or has been deleted}
        \lineii{S02}{Attempt to access an instance that is no longer
            valid or has been deleted}
        \lineii{S03}{Clear operation on an environment has failed}
        \lineii{S04}{Attempt to access an environment that has been
            deleted}
        \lineii{S05}{Attempt to operate on alias of current environment}
        \lineii{S06}{Attempt to create more environments than allowed}
        \lineii{S00}{Generic internal system error}
        \lineii{R01}{The logical buffer (I/O Stream) has been misused}
	\lineii{R02}{The logical buffer with given name could not be
            found}
	\lineii{R03}{Could not write to a read-only logical buffer}
\end{tableii}

These codes can be extracted from the exception description and used to
determine errors -- for instance, in an \code{if ... elif ...} control
statement. Some of these errors are caught by the \pyclips{} high-level
layer and interpreted in different ways (e.g. the \code{C13} error is
used to generate lists or to return \constant{None} after last element
in a list).

There are also some CLIPS specific exceptions that can be thrown at the
higher level: they are identified by a code beginning with the letter
\code{M}. A list of these exceptions follows, along with their
description:

\begin{tableii}{l|l}{code}{Code}{Description}
        \lineii{M01}{A constructor could not create an object}
        \lineii{M02}{An object could not be found in the CLIPS subsystem}
        \lineii{M03}{An attempt was made to \function{pickle} an object}
        \lineii{M99}{Wrong Python version, module could not initialize}
\end{tableii}

Finally, there is a particular error that occurs in case of \emph{fatal}
memory allocation failures, which is identified by a particular
exception, namely \exception{ClipsMemoryError}. This excepion is raised
with the following code and has the following meaning:

\begin{tableii}{l|l}{code}{Code}{Description}
        \lineii{X01}{The CLIPS subsystem could not allocate memory}
\end{tableii}

In this case the calling program \emph{must} exit, as the underlying
engine has reached an unstable state. An exception different from the
standard \exception{ClipsError} has been provided in order to allow
quick and effective countermeasures\footnote{Please note that in some
cases, and depending on how the operating system treats memory
allocation failures, the Python interpreter too could loose stability
in case of memory shortage.}.



\chapter{Multithreading\label{pyclips-threading}}

The CLIPS engine is a separate subsystem, as stated many times before.
In other words, it maintains a state independently from what happens in
the Python interpreter. This also means that, since CLIPS was never
conceived to be used as a multithreaded library, multiple threads should
not try to access the engine concurrently. The Python interpreter, due
to its nature, does not actually allow concurrent calls to the low-level
module, so it is safe to create concurrent threads that interact with
\pyclips{}.

However this has the side effect that, during a time consuming task
(such as calling \function{Run()} on a complex set of rules and facts)
the calling thread may block the other ones.

A partial solution to this, to allow multiple threads to switch more
reactively, is to call \function{Run()} with the \var{limit} parameter,
which specifies the number of rules to be fired at once. Of course this
allows subsequent calls to the CLIPS engine to modify its state, and
consequently enables execution paths that could be different from the
ones that a full CLIPS \code{(run)} would normally cause. Obviously
this only applies to the \function{Run()} function.

The consideration also implies that multithreaded Python applications
should take care of examining the engine state before interacting with
it, especially when splitting \function{Run()} (which normally modifies
the state) in multiple calls.



\chapter{Missing Features\label{pyclips-missing}}

Most of the CLIPS API is implemented in \pyclips{}. The \emph{lower
level} interface (which directly maps CLIPS exposed functions as
described in \clipsapg{}) can be accessed using the
\module{clips._clips} submodule. Almost all the functions defined here
have a counterpart in the CLIPS API, and a combined use of documentation
strings and \clipsapg{} itself can allow you to directly manipulate the
CLIPS subsystem as you would have done by embedding it in a C program.
However, there are some functions that appear in \code{dir(clips._clips)}
but, when called, issue an error:

\begin{verbatim}
>>> clips._clips.addClearFunction()
Traceback (most recent call last):
  File "<pyshell#46>", line 1, in ?
    clips._clips.addClearFunction()
ClipsError: C98: unimplemented feature/function
\end{verbatim}

and in fact, even their documentation string reports

\begin{verbatim}
>>> print clips._clips.addClearFunction.__doc__
unimplemented feature/function
\end{verbatim}

even if the name of the function is defined and is a \emph{callable}.

A list of such functions follows:

\begin{tableii}{l|l}{code}{Function}{Type}
	\lineii{addClearFunction}{execution hook}
	\lineii{addPeriodicFunction}{execution hook}
	\lineii{addResetFunction}{execution hook}
	\lineii{removeClearFunction}{execution hook}
	\lineii{removePeriodicFunction}{execution hook}
	\lineii{removeResetFunction}{execution hook}
	\lineii{addRunFunction}{execution hook}
	\lineii{removeRunFunction}{execution hook}
	\lineii{decrementFactCount}{reference count handler}
	\lineii{incrementFactCount}{reference count handler}
	\lineii{decrementInstanceCount}{reference count handler}
	\lineii{incrementInstanceCount}{reference count handler}
	\lineii{setOutOfMemoryFunction}{memory handler hook}
	\lineii{addEnvironmentCleanupFunction}{execution hook}
	\lineii{allocateEnvironmentData}{memory handler}
	\lineii{deallocateEnvironmentData}{memory handler}
	\lineii{destroyEnvironment}{environment destructor}
	\lineii{getEnvironmentData}{memory handler}
\end{tableii}

The description of these functions is outside the scope of this guide,
and can be found in \clipsapg{} of CLIPS. It is not likely that these
functions will be implemented even in future versions of \pyclips{} since
Python programmers are usually not interested in dealing with low level
memory handling (which is the primary use of the memory oriented
functions), and tasks like reference count handling are performed
directly by Python itself (for Python objects which shadow CLIPS
entities) and by the low level \pyclips{} submodule. Also, the functions
referred to above as \emph{execution hooks} often have to deal with CLIPS
internal structures at a very low level, so they would be of no use in a
Python program.

Other API functions, which are used internally by \pyclips{} (for
instance the \function{InitializeEnvironment()} C function), are not
implemented in the module.

\begin{notice}
Some of the features (either in current and possibly in further versions
of \pyclips{}) may depend on the version of CLIPS that is used to compile
the module. Using the most recent stable version of CLIPS is recommended
in order to enable all \pyclips{} features. Features that are excluded
from the module because of this reason will issue an exception, in which
the exception text reports the following: \code{"C97: higher engine
version required"}. Moreover, the CLIPS engine version may affect the
behaviour of some functions. Please consider reading the documentation
related to the used CLIPS version when a function does not behave as
expected.
\end{notice}


\chapter{Installing \pyclips{}\label{pyclips-setup}}

\section{Installation\label{pyclips-setup-installation}}

To install \pyclips{} you should also download the full CLIPS source
distribution. You will find a file called \file{CLIPSsrc.zip} at the
CLIPS download location: you should choose to download this instead of
the \UNIX{} compressed source, since the setup program itself performs
the task of extracting the files to an appropriate directory with the
correct line endings. The ZIP file format has been chosen in order to
avoid using different extraction methods depending on the host operating
system.

\pyclips{} uses \module{distutils} or \module{setuptools} for its
installation. So in all supported systems the module can be easily
set up once the whole source has been extracted to a directory and
the CLIPS source code has been put in the same place, by means of the
following command:

\begin{verbatim}
# python setup.py install
\end{verbatim}

In fact recent versions of \pyclips{} will attempt to download the
latest supported CLIPS source directly from the \pyclips{} web site
if no CLIPS source package is found. Otherwise no attempt to connect
to the Internet will be made. The \file{README} file provides more
up-to-date and detailed information on the setup process.

On \UNIX{}, if you have a system-wide Python distribution, your
privileges for installation should be the same as the Python owner.

The CLIPS library itself is compiled for a specific platform, since
\file{setup.py} modifies the \file{setup.h} file in the CLIPS
distribution. 

\pyclips{} is known to build and pass the tests on \emph{Linux (x86 and
x86_64)}\footnote{The x86_64 platforms requires some optional patches
to be applied.}, \emph{Win32} (many flavours of it have been tested),
\emph{Sun Solaris} with 32-bit gcc, \emph{FreeBSD}, \emph{Mac OS X} with
\emph{Fink} and, using a customized build process, has been ported to the
\emph{Sharp Zaurus (SA-1110)} platform.


\section{Requirements\label{pyclips-setup-requirements}}

\pyclips{} requires Python 2.4 or above to function: it uses decorators
to check and enforce types where needed, and in some places it also uses
modern aspects of the Python API.

At least version 6.23 of CLIPS is required: it allows the definition and
use of \emph{environments}, and the function and macro definitions are
more conformant to the ones described in \clipsapg{}. Of course features
present in CLIPS 6.24 are not available when using the previous CLIPS
version, so if there is no particular reason to use it, please compile
PyCLIPS with CLIPS 6.24, which also fixes some bugs.

